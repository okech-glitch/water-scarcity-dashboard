\documentclass[a4paper,12pt]{article}
\usepackage[utf8]{inputenc}
\usepackage[T1]{fontenc}
\usepackage{lmodern}
\usepackage{geometry}
\geometry{margin=1in}
\usepackage{graphicx}
\usepackage{booktabs}
\usepackage{caption}

% Preamble for document structure and packages
\title{Water Scarcity Impact Report}
\author{AI Water Scarcity Team}
\date{August 2025}

\begin{document}

\maketitle

\section{Introduction}
This report outlines the impact of the AI-powered Water Scarcity Dashboard in Kenya, designed to enhance agricultural resilience and food security.

\section{Impact Metrics}
\begin{itemize}
    \item \textbf{Current Yield}: 100,000 tonnes, feeding 0.67 million people.
    \item \textbf{Projected Yield (+20\%)}: 120,000 tonnes, feeding 0.80 million people.
    \item \textbf{Projected Yield (+30\%)}: 130,000 tonnes, feeding 0.87 million people.
\end{itemize}

\section{Benefits}
\begin{itemize}
    \item Yield increase of up to 30\% with AI adoption.
    \item Enhanced food security for millions through optimized water use.
    \item Economic savings of 12.5–18.75 billion KES in drought losses.
    \item Alignment with SDG 13 (Climate Action), SDG 2 (Zero Hunger), and SDG 6 (Clean Water).
\end{itemize}

\section{Future Potential}
Scalable across Africa, where 65\% of agriculture is rainfed, this tool can be deployed to millions of farmers through partnerships with NGOs and governments.

\end{document}